%        File: Notes.tex
%     Created: Tue Jun 13 04:00 pm 2017 B
% Last Change: Tue Jun 13 04:00 pm 2017 B
%
\documentclass[twocolumn]{report}

\usepackage{titling}
\usepackage{amsmath}

\title{Discrete Mathmatics}
\author{Sina Aleyaasin}

\begin{document}
\begin{titlingpage}
	\maketitle
	\begin{abstract}
		A compilation of notes based on \textit{Discrete Mathematics and its Applications} ($7^{th}$ Edition) -- K. Rosen. 
	\end{abstract}
\end{titlingpage}
\tableofcontents
\chapter{Logic and Proofs}
\section{Propositional Logic}
\subsection{Propositions}
A proposition is a declarative sentence that has a truth value of true (T) or false (F). 
It may be denoted by a propositional variable (such as $p, q, r, s$). 
Compound propositions can be made from one or more propositions by means of \textit{logical connectives}. 
The definition of a logical connective can be expressed by its corresponding truth table.

\subsection{Logical Connectives}
A logical connective operates on one or more propositions to yield a new proposition.

\subsubsection{Negation $\neg$}
Negation is a unary operation that yields the logical complement of a proposition.
\begin{table}[h]
\centering
\label{tab: negation-truth-table}
\begin{tabular}{cc}
$p$ & $\neg p$ \\ \hline
T & F     \\
F & T    
\end{tabular}
\caption{Truth table for negation}
\end{table}

\subsubsection{Conjuction $\wedge$}
The conjuction of two propositions is true if both propositions are true, and is false otherwise.
\begin{table}[h]
	\centering
	\label{tab: conjunction-truth-table}
	\begin{tabular}{cc|c}
		$p$ & $q$ & $p\wedge q$ \\ \hline
		T & T & T \\
		T & F & F \\
		F & T & F \\
		F & F & F \\
	\end{tabular}
	\caption{Truth table for conjunction}
\end{table}

\subsubsection{Disjunction $\lor$}
The disjunction of two propositions is true if either or both are true, and is false otherwise.
\begin{table}[h]
	\centering
	\label{tab: disjunction-truth-table}
	\begin{tabular}{cc|c}
		$p$ & $q$ & $p\lor q$ \\ \hline
		T & T & T \\
		T & F & T \\
		F & T & T \\
		F & F & F \\
	\end{tabular}
	\caption{Truth table for disjunction}
\end{table}

\subsubsection{Exclusive Disjunction $\oplus$}
The exclusive disjunction of two propositions is true if one is the logical complement of the other, and is false otherwise. 
\begin{table}[h]
	\centering
	\label{tab: exclusive-disjuction-truth-table}
	\begin{tabular}{cc|c}
		$p$ & $q$ & $p\oplus q$ \\ \hline
		T & T & F \\
		T & F & T \\
		F & T & T \\
		F & F & F \\
	\end{tabular}
	\caption{Truth table for exclusive disjunction}
\end{table}

\subsection{Conditional Statements}
A conditional statement is a proposition composed from a \textit{premise} $p$ and a \textit{conclusion} $q$. 
It represents the implication of $q$ by $p$, and is denoted 
\[
	p \rightarrow q
\]
The only case in which this statement is false is when $p$ is true but $q$ is false. 
\begin{table}[h]
	\centering
	\label{tab: conditional-statement-truth-table}
	\begin{tabular}{cc|c}
		$p$ & $q$ & $p \rightarrow q$ \\ \hline
		T & T & T \\
		T & F & F \\
		F & T & T \\
		F & F & T \\
	\end{tabular}
	\caption{Truth table for a conditional statement}
\end{table}

\subsubsection{Converse}
The converse of a conditional statement $p \rightarrow q$ is given by
\[
	q \rightarrow p
\]

\subsubsection{Contrapositive}
The contrapositive of a conditional statement is given by 
\[
	\neg p \rightarrow \neg q 
\]
It is \textit{logically equivelant} to the conditional statement itself; that is to say, the truth tables for a conditional statement and its contrapositvie are \underline{identical}. 

\subsubsection{Inverse}
The inverse of a conditional statement is given by
\[
	\neg q \rightarrow \neg p
\]
Since a conditional statement is logically equivelant to its contrapositive, and the inverse is the contrapositive of the converse, then the inverse is thus logically equivelant to the converse.

\subsubsection{Biconditional Statements}
A biconditional statement is the conjucntion of a contitional statement and its converse. It is denoted by 
\[
	p \leftrightarrow q \equiv (p \rightarrow q) \wedge (q \rightarrow p)
\]
and is only true when both $p$ and $q$ have the same truth value. 
\begin{table}[h]
	\centering
	\label{tab: biconditional-truth-table}
	\begin{tabular}{cc|c}
		$p$ & $q$ & $p \leftrightarrow q$ \\ \hline
		T & T & T \\
		T & F & F \\
		F & T & F \\
		F & F & T \\
	\end{tabular}
	\caption{Truth table for a biconditional statement.}
\end{table}

\subsection{Logical \& Bit operations}
A \textit{bit} (portamentau of binary and digit) is the smallest unit of information representable by a computer. 
The value of a bit can be either $1$ or $0$, analagous to the truth value of a propositon, T or F, respectively.
One may consider bit operators analogous to logical operators. 
\begin{table}[h]
	\centering
	\label{tab: bit-operators}
	\begin{tabular}{c|c}
		Logical Connective & Bit Operator \\ \hline
		$\lor$ & OR \\
		$\wedge$ & AND \\
		$\oplus$ & XOR \\
	\end{tabular}
	\caption{Correspondence between logical connectives and bit operators}
\end{table}

\section{Applications of Propositional Logic}
\subsection{Logic Gates}

\section{Propositional Equivelances}
\subsection{Compound Propositions}
A compound proposition may be categorised into one of three categories:
\subsubsection{Tautology}
A \textit{tautology} is a compound proposition that is \textit{true} in all possible cases. 
An arbitrary tautology is denoted by 
\[
	\top
\]
\subsubsection{Contradiction}
A \textit{contradiction} is a compound propsition that \textit{false} in all possible cases.
An arbitrary contradiction is denoted by
\[
	\bot
\]
\subsubsection{Contingent}
A \textit{contingent} proposition is one that is neither a tautology nor a contradiction. 
Its truth value is contingent on the particular configuration of truth values of its constituent propositions.

\subsection{Logical Equivalences}
The compound propositions $p$ and $q$ are \textit{logically equivelant} if 
\begin{equation}
	p \leftrightarrow q \equiv \top
	\label{eqn: logical-equivelance}
\end{equation}
This statement may be equally expressed by 
\[
	p \equiv q
\]
The following are important logical equivelances:
\subsubsection{Domination Laws}
\begin{subequations}
	\begin{equation}
		p \lor \top \equiv \top
		\label{eqn: disjunction-domination-law}
	\end{equation}
	\begin{equation}
		p \wedge \bot \equiv \bot
		\label{eqn: conjunction-domination-law}
	\end{equation}
\end{subequations}
\subsubsection{Identity Laws}
\begin{subequations}
\begin{equation}
	p \wedge \top \equiv p
	\label{eqn: conjunction-identity-law}
\end{equation}
\begin{equation}
	p \lor \bot \equiv p
	\label{eqn: disjuction-identity-law}
\end{equation}
\end{subequations}
\subsubsection{Negation laws}
\begin{subequations}
	\begin{equation}
		p \wedge \neg p \equiv \bot
		\label{eqn: conjunction-negation-law}
	\end{equation}
	\begin{equation}
		p \lor \neg p \equiv \top
		\label{eqn: disjunction-negation-law}
	\end{equation}
\end{subequations}
\subsubsection{Double Negation Law}
\begin{subequations}
	\begin{equation}
		\neg (\neg p) \equiv p
		\label{eqn: double-negation-law}
	\end{equation}
\end{subequations}
\subsubsection{Commutative Laws}
The commutativity of a binary logical connective can be determined by examining its operation on two propositions with complementary truth values. 
From Table \ref{tab: conjunction-truth-table}, one observes that
\[
	\top \wedge \bot \equiv \bot \wedge \top \equiv \bot
\]
Likewise, from Table \ref{tab: disjunction-truth-table}
\[
	\top \lor \bot \equiv \bot \lor \top \equiv \top
\]
This leads to the \textit{commutative laws}
\begin{subequations}
	\begin{equation}
		p \wedge q \equiv q \wedge p
		\label{eqn: conjunction-commutative-law}
	\end{equation}
	\begin{equation}
		p \lor q \equiv q \lor p
		\label{eqn: disjunction-commutative-law}
	\end{equation}
\end{subequations}
A similar argument may be used to demonstrate the commutativity of $\oplus$ and $\leftrightarrow$.
\subsubsection{Associative Laws}
The associativity of $\lor$ and $\wedge$ can be verified by examining Tables \ref{tab: disjunction-associativity-truth-table} and \ref{tab: conjunction-associativity-truth-table} respectfully.
\begin{table}[h]
	\centering
	\begin{tabular}{ccc|c|c|c|c}
		$p$ & $q$ & $r$ & $(p\lor q)$ & $(q \lor r)$ & $(p \lor q) \lor r$ & $p \lor (q \lor r)$ \\ \hline
		T & T & T & T & T & \textbf{T} & \textbf{T}  \\
		T & T & F & T & T & \textbf{T} & \textbf{T}  \\
		T & F & T & T & T & \textbf{T} & \textbf{T}  \\
		T & F & F & T & F & \textbf{T} & \textbf{T}  \\
		F & T & T & T & T & \textbf{T} & \textbf{T}  \\
		F & T & F & T & T & \textbf{T} & \textbf{T}  \\
		F & F & T & F & T & \textbf{T} & \textbf{T}  \\
		F & F & F & F & F & \textbf{F} & \textbf{F}  \\
	\end{tabular}
	\caption{Associativity of $\lor$ by truth table.}
	\label{tab: disjunction-associativity-truth-table}
\end{table}
\begin{table}[h]
	\centering
	\begin{tabular}{ccc|c|c|c|c}
		$p$ & $q$ & $r$ & $(p\wedge q)$ & $(q \wedge r)$ & $(p \wedge q) \wedge r$ & $p \wedge (q \wedge r)$ \\ \hline
		T & T & T & T & T & \textbf{T} & \textbf{T}  \\
		T & T & F & T & F & \textbf{F} & \textbf{F}  \\
		T & F & T & F & F & \textbf{F} & \textbf{F}  \\
		T & F & F & F & F & \textbf{F} & \textbf{F}  \\
		F & T & T & F & T & \textbf{F} & \textbf{F}  \\
		F & T & F & F & F & \textbf{F} & \textbf{F}  \\
		F & F & T & F & F & \textbf{F} & \textbf{F}  \\
		F & F & F & F & F & \textbf{F} & \textbf{F}  \\
	\end{tabular}
	\caption{Associativity of $\wedge$ by truth table.}
	\label{tab: conjunction-associativity-truth-table}
\end{table}

Hence
\begin{subequation}
	\begin{equation}
		p \lor (q \lor r) \equiv (p \lor q) \lor r
		\label{eqn: disjunction-associative-law}
	\end{equation}
	\begin{equation}
		p \wedge (q \wedge r) \equiv (p\wedge q) \wedge r
		\label{eqn: conjunction-associative-law}
	\end{equation}
\end{subequation}

\subsubsection{Distributive Laws}
The distributivity of $\lor$ and $\wedge$ is demonstrated in Tables \ref{tab: disjunction-distributivity-truth-table} and \ref{tab: conjunction-distributivity-truth-table} respectfully.
\begin{table}[h]
	\centering
	\begin{tabular}{ccc|c|c|c}
		$p$ & $q$ & $r$ & $(p \lor r)$ & $p \lor (q \wedge r) $ & $(p \lor q) \wedge (p \lor r)$ \\ \hline
		T & T & T & T & \textbf{T} & \textbf{T}  \\
		T & T & F & T & \textbf{T} & \textbf{T}  \\
		T & F & T & T & \textbf{T} & \textbf{T}  \\
		T & F & F & T & \textbf{T} & \textbf{T}  \\
		F & T & T & T & \textbf{T} & \textbf{T}  \\
		F & T & F & F & \textbf{F} & \textbf{F}  \\
		F & F & T & T & \textbf{F} & \textbf{F}  \\
		F & F & F & F & \textbf{F} & \textbf{F}  \\
	\end{tabular}
	\caption{Distributivity of $\lor$ by truth table. Columns of $(p\lor q)$ and $(q \wedge r)$ are borrowed from Tables \ref{tab: disjunction-associativity-truth-table} and \ref{tab: conjunction-associativity-truth-table} respectively.}
	\label{tab: disjunction-distributivity-truth-table}
\end{table}


\begin{table}[h]
	\centering
	\begin{tabular}{ccc|c|c|c}
		$p$ & $q$ & $r$ & $(p \wedge r)$ & $p \wedge (q \lor r) $ & $(p \wedge q) \lor (p \wedge r)$ \\ \hline
		T & T & T & T & \textbf{T} & \textbf{T}  \\
		T & T & F & F & \textbf{T} & \textbf{T}  \\
		T & F & T & T & \textbf{T} & \textbf{T}  \\
		T & F & F & F & \textbf{F} & \textbf{F}  \\
		F & T & T & F & \textbf{F} & \textbf{F}  \\
		F & T & F & F & \textbf{F} & \textbf{F}  \\
		F & F & T & F & \textbf{F} & \textbf{F}  \\
		F & F & F & F & \textbf{F} & \textbf{F}  \\
	\end{tabular}
	\caption{Distributivity of $\wedge$ by truth table. Columns of $(q\lor r)$ and $(p \wedge q)$ are borrowed from Tables \ref{tab: disjunction-associativity-truth-table} and \ref{tab: conjunction-associativity-truth-table} respectively.}
	\label{tab: conjunction-distributivity-truth-table}
\end{table}

\subsubsection{De Morgan's Laws}
De Morgan's laws demonstrate how conjunctions and disjunctions are negated. 
They take the form
\begin{subequations}
	\begin{equation}
		\neg (p\wedge q) \equiv \neg p \lor \neg q
		\label{eqn: negation-of-conjunction}
	\end{equation}
	\begin{equation}
		\neg (p\lor q) \equiv \neg p \wedge \neg q
		\label{eqn: negation-of-disjunction}
	\end{equation}
\end{subequations}
DO TRUTH TABLES
\subsubsection{Absorption Laws}
DO TRUTH TABLES
\subsubsection{Logical Equivelances with $\rightarrow$}
DO TABLE
\subsubsection{Logical Equivelances with $\leftrightarrow$}
DO TABLE


\end{document}



