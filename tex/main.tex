%        File: Notes.tex
%     Created: Tue Jun 13 04:00 pm 2017 B
% Last Change: Tue Jun 13 04:00 pm 2017 B
%
\documentclass[twocolumn]{report}

\usepackage{titling}
\usepackage{amsmath}

\title{Discrete Mathmatics}
\author{Sina Aleyaasin}

\begin{document}
\begin{titlingpage}
	\maketitle
	\begin{abstract}
		A compilation of notes based on \textit{Discrete Mathematics and its Applications} ($7^{th}$ Edition) -- K. Rosen. 
	\end{abstract}
\end{titlingpage}
\tableofcontents
\chapter{Logic and Proofs}
\section{Propositional Logic}
\subsection{Propositions}
A proposition is a declarative sentence that has a truth value of true (T) or false (F). 
It may be denoted by a propositional variable (such as $p, q, r, s$). 
Compound propositions can be made from one or more propositions by means of \textit{logical connectives}. 
The definition of a logical connective can be expressed by its corresponding truth table.

\subsection{Logical Connectives}
A logical connective operates on one or more propositions to yield a new proposition.

\subsubsection{Negation $\neg$}
Negation is a unary operation that yields the logical complement of a proposition.
\begin{table}[h]
\centering
\label{tab: negation-truth-table}
\begin{tabular}{cc}
$p$ & $\neg p$ \\ \hline
T & F     \\
F & T    
\end{tabular}
\caption{Truth table for negation}
\end{table}

\subsubsection{Conjuction $\wedge$}
The conjuction of two propositions is true if both propositions are true, and is false otherwise.
\begin{table}[h]
	\centering
	\label{tab: conjunction-truth-table}
	\begin{tabular}{cc|c}
		$p$ & $q$ & $p\wedge q$ \\ \hline
		T & T & T \\
		T & F & F \\
		F & T & F \\
		F & F & F \\
	\end{tabular}
	\caption{Truth table for conjunction}
\end{table}

\subsubsection{Disjunction $\lor$}
The disjunction of two propositions is true if either or both are true, and is false otherwise.
\begin{table}[h]
	\centering
	\label{tab: disjunction-truth-table}
	\begin{tabular}{cc|c}
		$p$ & $q$ & $p\lor q$ \\ \hline
		T & T & T \\
		T & F & T \\
		F & T & T \\
		F & F & F \\
	\end{tabular}
	\caption{Truth table for disjunction}
\end{table}

\subsubsection{Exclusive Disjunction $\oplus$}
The exclusive disjunction of two propositions is true if one is the logical complement of the other, and is false otherwise. 
\begin{table}[h]
	\centering
	\label{tab: exclusive-disjuction-truth-table}
	\begin{tabular}{cc|c}
		$p$ & $q$ & $p\oplus q$ \\ \hline
		T & T & F \\
		T & F & T \\
		F & T & T \\
		F & F & F \\
	\end{tabular}
	\caption{Truth table for exclusive disjunction}
\end{table}

\subsection{Conditional Statements}
A conditional statement is a proposition composed from a \textit{premise} $p$ and a \textit{conclusion} $q$. 
It represents the implication of $q$ by $p$, and is denoted 
\[
	p \rightarrow q
\]
The only case in which this statement is false is when $p$ is true but $q$ is false. 
\begin{table}[h]
	\centering
	\label{tab: conditional-statement-truth-table}
	\begin{tabular}{cc|c}
		$p$ & $q$ & $p \rightarrow q$ \\ \hline
		T & T & T \\
		T & F & F \\
		F & T & T \\
		F & F & T \\
	\end{tabular}
	\caption{Truth table for a conditional statement}
\end{table}

\subsubsection{Converse}
The converse of a conditional statement $p \rightarrow q$ is given by
\[
	q \rightarrow p
\]

\subsubsection{Contrapositive}
The contrapositive of a conditional statement is given by 
\[
	\neg q \rightarrow \neg p 
\]
It is \textit{logically equivelant} to the conditional statement itself; that is to say, the truth tables for a conditional statement and its contrapositvie are \underline{identical}. 

\subsubsection{Inverse}
The inverse of a conditional statement is given by
\[
	\neg p \rightarrow \neg q
\]
Since a conditional statement is logically equivelant to its contrapositive, and the inverse is the contrapositive of the converse, then the inverse is thus logically equivelant to the converse.

\subsubsection{Biconditional Statements}
A biconditional statement is the conjucntion of a contitional statement and its converse. It is denoted by 
\[
	p \leftrightarrow q \equiv (p \rightarrow q) \wedge (q \rightarrow p)
\]
and is only true when both $p$ and $q$ have the same truth value. 
\begin{table}[h]
	\centering
	\label{tab: biconditional-truth-table}
	\begin{tabular}{cc|c}
		$p$ & $q$ & $p \leftrightarrow q$ \\ \hline
		T & T & T \\
		T & F & F \\
		F & T & F \\
		F & F & T \\
	\end{tabular}
	\caption{Truth table for a biconditional statement.}
\end{table}

\subsection{Logical \& Bit operations}
A \textit{bit} (portamentau of binary and digit) is the smallest unit of information representable by a computer. 
The value of a bit can be either $1$ or $0$, analagous to the truth value of a propositon, T or F, respectively.
One may consider bit operators analogous to logical operators. 
\begin{table}[h]
	\centering
	\label{tab: bit-operators}
	\begin{tabular}{c|c}
		Logical Connective & Bit Operator \\ \hline
		$\lor$ & OR \\
		$\wedge$ & AND \\
		$\oplus$ & XOR \\
	\end{tabular}
	\caption{Correspondence between logical connectives and bit operators}
\end{table}

\section{Applications of Propositional Logic}
\subsection{Logic Gates}

\section{Propositional Equivelances}
\subsection{Compound Propositions}
A compound proposition may be categorised into one of three categories:
\subsubsection{Tautology}
A \textit{tautology} is a compound proposition that is \textit{true} in all possible cases. 
An arbitrary tautology is denoted by 
\[
	\top
\]
\subsubsection{Contradiction}
A \textit{contradiction} is a compound propsition that \textit{false} in all possible cases.
An arbitrary contradiction is denoted by
\[
	\bot
\]
\subsubsection{Contingent}
A \textit{contingent} proposition is one that is neither a tautology nor a contradiction. 
Its truth value is contingent on the particular configuration of truth values of its constituent propositions.

\subsection{Logical Equivalences}
The compound propositions $p$ and $q$ are \textit{logically equivelant} if 
\begin{equation}
	p \leftrightarrow q \equiv \top
	\label{eqn: logical-equivelance}
\end{equation}
This statement may be equally expressed by 
\[
	p \equiv q
\]
The following are important logical equivelances:
\subsubsection{Domination Laws}
Disjunction with a tautology is a tautology, and conjunction with a contradiction is a contradiction
\begin{subequations}
	\begin{equation}
		p \lor \top \equiv \top
		\label{eqn: disjunction-domination-law}
	\end{equation}
	\begin{equation}
		p \wedge \bot \equiv \bot
		\label{eqn: conjunction-domination-law}
	\end{equation}
\end{subequations}
\subsubsection{Identity Laws}
Conjunction with a tautology, and disjunction with a contradiction, are \textit{identity} operations on a proposition 
\begin{subequations}
\begin{equation}
	p \wedge \top \equiv p
	\label{eqn: conjunction-identity-law}
\end{equation}
\begin{equation}
	p \lor \bot \equiv p
	\label{eqn: disjuction-identity-law}
\end{equation}
\end{subequations}
\subsubsection{Negation laws}
Conjunction of a proposition with its negation is a contradiction, and disjunction of a proposition with its negation is a tautology
\begin{subequations}
	\begin{equation}
		p \wedge \neg p \equiv \bot
		\label{eqn: conjunction-negation-law}
	\end{equation}
	\begin{equation}
		p \lor \neg p \equiv \top
		\label{eqn: disjunction-negation-law}
	\end{equation}
\end{subequations}
\subsubsection{Double Negation Law}
A proposition is the negation of its negation
\begin{subequations}
	\begin{equation}
		\neg (\neg p) \equiv p
		\label{eqn: double-negation-law}
	\end{equation}
\end{subequations}
\subsubsection{Idempotent Laws}
Both the conjunction and disjunction of a proposition with itself yields itself
\begin{subequations}
	\begin{equation}
		p \wedge p \equiv p
		\label{eqn: idempotent-conjunction}
	\end{equation}
	\begin{equation}
		p \lor p \equiv p
		\label{eqn: idempotent-disjunction}
	\end{equation}
\end{subequations}
\subsubsection{Absorption Laws}
The former two equivelances may be \textit{absorbed} into each other as in the following 
\begin{subequations}
	\begin{equation}
		p \wedge (p \lor p) \equiv p
		\label{eqn: absorption-1}
	\end{equation}
	\begin{equation}
		p \lor (p \wedge p) \equiv p
		\label{eqn: absorption-2}
	\end{equation}
\end{subequations}
\subsubsection{Commutative Laws}
The commutativity of a binary logical connective can be determined by examining its operation on two propositions with complementary truth values. 
From Table \ref{tab: conjunction-truth-table}, one observes that
\[
	\top \wedge \bot \equiv \bot \wedge \top \equiv \bot
\]
Likewise, from Table \ref{tab: disjunction-truth-table}
\[
	\top \lor \bot \equiv \bot \lor \top \equiv \top
\]
This leads to the \textit{commutative laws}
\begin{subequations}
	\begin{equation}
		p \wedge q \equiv q \wedge p
		\label{eqn: conjunction-commutative-law}
	\end{equation}
	\begin{equation}
		p \lor q \equiv q \lor p
		\label{eqn: disjunction-commutative-law}
	\end{equation}
\end{subequations}
A similar argument may be used to demonstrate the commutativity of $\oplus$ and $\leftrightarrow$.
\subsubsection{Associative Laws}
Both conjunction and disjunction are \textit{associative}
\begin{subequations}
	\label{eqn: associative-laws}
	\begin{equation}
		p \lor (q \lor r) \equiv (p \lor q) \lor r
		\label{eqn: disjunction-associative-law}
	\end{equation}
	\begin{equation}
		p \wedge (q \wedge r) \equiv (p\wedge q) \wedge r
		\label{eqn: conjunction-associative-law}
	\end{equation}
\end{subequations}

\subsubsection{Distributive Laws}
Conjunction and disjunction \textit{distribute} as in the following 
\begin{subequations}
	\begin{equation}
		p\wedge(q\lor r)\equiv(p\wedge q)\lor(p\wedge r)
		\label{eqn: conjunction-distributivity}
	\end{equation}
	\begin{equation}
		p\lor(q\wedge r)\equiv(p\lor q)\wedge(p\lor r)
		\label{eqn: disjunction-distributivity}
	\end{equation}
	\label{eqn: distributive-laws}
\end{subequations}
\subsubsection{De Morgan's Laws}
De Morgan's laws demonstrate how conjunctions and disjunctions are negated
\begin{subequations}
	\label{eqn: demorgan-laws}
	\begin{equation}
		\neg (p\wedge q) \equiv \neg p \lor \neg q
		\label{eqn: negation-of-conjunction}
	\end{equation}
	\begin{equation}
		\neg (p\lor q) \equiv \neg p \wedge \neg q
		\label{eqn: negation-of-disjunction}
	\end{equation}
\end{subequations}
\subsubsection{Logical Equivelances with $\rightarrow$}
DO TABLE
\subsubsection{Logical Equivelances with $\leftrightarrow$}
DO TABLE

\subsection{Propositional Satisfiability}
Consider the set of $N$ propositions 
\[
	\{s_{1}, s_{2}, \dots, s_{N}\} = S_{N}
\]
There exist $2^{N}$ possible configurations of truth values for the propositions contained in $S_{N}$.
Let $p$ be a compound proposition that is some amalgamation of the propositions in $S_{N}$ through logical connectives. 
We say that $p$ is \textit{satisfiable} if there exists a configuration of truth values in $S_{N}$ such that $p$ is true.
We call this configuration a \textit{solution} of $p$.
If there exist no solution to $p$, we say the proposition is \textit{unsatisfiable}.
That is, a proposition is unsatisfiable if and only if its negation is a tautology. 

\section{Predicates and Quantifiers}
\subsection{Predicates}
A \textit{predicate} is statement regarding a \textit{subject} that becomes a proposition when the subject is specified.
A predicate may be condisered as a \textit{propositional function}; that is, a mapping from the subject to a bivalent truth value.
We denote a unary predicate as 
\[
	P(x) 
\]
where $P$ is the predicate condition, and $x$ is a variable that denotes the subject.
A predecate may have multiple subjects; the predicate 
\[
	P(x_{1}, x_{2}, \dots, x_{n})
\]
is refered to as an \textit{n-ary} predicate.

\subsection{Quantifiers}
The variable $x$ in the statement $P(x)$ is a \textit{free} variable.
In the previous section we discussed how $P(x)$ becomes a proposotion when $x$ is set to a value.
Here, we discuss another way in which $P(x)$ becomes a proposition: \textit{binding} $x$ with a \textit{universal quantifier}.
\subsubsection{Universal Quantifier}
Consider the set 
\[
	D = \{x_{1}, x_{2}, \dots, x_{n}\}
\]
which we shall call the \textit{domain of discourse}.
One can construct a compound proposition that is the conjunction of the predicate $P(x)$ for all $x$ in $D$. 
That is 
\[
	P(x_{1}) \wedge P(x_{2}) \wedge \dots \wedge P(x_{n}) \equiv \bigwedge_{i=1}^{n}P(x_{i})
\]
Such a prescription is sufficient when the domain of dicource is \textit{finite}.
What is the equivelent proposition if $D$ is \textit{not} finite? 
\[
	\forall x P(x)
\]
where we use the \textbf{universal quantifier} $\forall$, read as ``for all''.

\subsubsection{Existential Quantifier}
Analagous to the universal quantifier, the proposition 
\[
	P(x_{1}) \lor P(x_{2}) \lor \dots \lor P(x_{n}) \equiv \bigvee_{i=1}^{n}P(x_{i})
\]
may be denoted for a non-finite universe of discourse by
\[
	\exists xP(x)
\]
where we use the \textbf{existential quantifier} $\exists$, read as ``there exists''.

\subsection{Negating Quantifiers}
Using the Associative Laws (\ref{eqn: associative-laws}), one may generalise De Morgan's Laws (\ref{eqn: demorgan-laws}) for $n$ propositions
\begin{subequations}
	\begin{equation}
		\neg \left(\bigwedge_{i = 1}^{n} p_{i}\right) \equiv \bigvee_{i=1}^{n}\neg p_{i}
		\label{eqn: general-demorgan-1}
	\end{equation}
	\begin{equation}
		\neg \left(\bigvee_{i = 1}^{n} p_{i}\right) \equiv \bigwedge_{i=1}^{n}\neg p_{i}
		\label{eqn: general-demorgan-2}
	\end{equation}
	\label{eqn: general-demorgan-laws}
\end{subequations}
Suppose  
\[
	p_{i} \equiv P(x_{i})
\]
for some predicate $P$ where the $x_{i}$ are members of a finite domain of discourse $D$.
What then if $D$ is \textit{not} finite?
\begin{subequations}
	\begin{equation}
		\neg \forall xP(x) \equiv \exists x\neg P(x)
		\label{eqn: univeral-quantifier-negation}
	\end{equation}
	\begin{equation}
		\neg \exists xP(x) \equiv \forall x\neg P(x)
		\label{eqn: existential-quantifier-negation}
	\end{equation}
	\label{eqn: demorgan-laws-quantifiers}
\end{subequations}
Equations \ref{eqn: demorgan-laws-quantifiers} are known as \textit{De Morgan's Laws for Quantifiers}.

\subsection{Nested Quantifiers}
Consider the $n$-ary predicate
\[
	P(x_{1}, x_{2}, \dots, x_{n})
\]
As discussed previously, this statement becomes a proposition if and only if all the free variables are either \textit{bound} or \textit{set} to a value.
Suppose we bind (or set) $s < n$ of the free variables. 
The result is  an ($s-n$)-ary predicate.
Thus, the act of \textit{nesting} quantifiers is a way in which a predicate can be made into either a proposition or a predicate of lower \textit{arity}. 
The total number of nested quantifiers may not exceed the arity of the predicate it quantifies.

\end{document}





